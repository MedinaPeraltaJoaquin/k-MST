\begin{algorithm}
\caption{Algoritmo de Optimización de Ballenas (Whale Optimization Algorithm - WOA)}
\label{alg:woa}
\begin{algorithmic}[1]
    \State Inicializar la población de ballenas $X_i$ ($i = 1, 2, \dots, n$)
    \State Calcular el fitness de cada agente de búsqueda
    \State $X^*$ = el mejor agente de búsqueda
    \While {$t < \text{número máximo de iteraciones}$}
        \For {cada agente de búsqueda}
            \State Actualizar $a$, $A$, $C$, $l$, y $p$
            \If {$p < 0.5$}
                \If {$|A| < 1$}
                    \State Actualizar la posición del agente de búsqueda actual por la Ec. (1)
                \Else
                    \State Seleccionar un agente de búsqueda aleatorio ($X_{\text{rand}}$)
                    \State Actualizar la posición del agente de búsqueda actual por la Ec. (3)
                \EndIf
            \Else
                \State Actualizar la posición del agente de búsqueda actual por la Ec. (5)
            \EndIf
        \EndFor
        \State Comprobar si algún agente de búsqueda sale del espacio de búsqueda y modificarlo
        \State Calcular el fitness de cada agente de búsqueda
        \State Actualizar $X^*$ si hay una mejor solución
        \State $t \leftarrow t+1$
    \EndWhile
    \State \Return $X^*$
\end{algorithmic}
\end{algorithm}

Donde las ecuaciones son:
\begin{align}
    X_t &= X^* - A \cdot D \\
    D &= |C\cdot X^* - X_t|
\end{align}

\begin{align}
    X &= X_{rand} - A \cdot D \\
    D &= |C \cdot X_{rand} - X_t|  
\end{align}

\begin{align}
    X_t &= D \cdot e^{bl} \cdot \cos{(2\pi l)} + X^* \\
    D &= |X^* - X_t|
\end{align}

Con el valor de las constantes como:
\begin{itemize}
    \item{
        $A = 2a \cdot r - a $
    }
    \item {
        $C = 2 \cdot r$
    }
    \item{
        $a \in [2,0]$ es un valor que va en decremento durante la cantidad de iteraciones.
    }
    \item {
        $r \in [0,1]$ es un valor aleatorio.
    }
    \item{
        $b$ es la constante definida de la forma logarítmica de la espiral.
    }
    \item{
        $l \in [-1,1]$ es un valor aleatorio.
    }
\end{itemize}

% Ecuación 20: Actualización binaria para BWOA 1 (Tangente Hiperbólica)
\begin{equation}
\label{eq:20}
\vec{\vartheta}(i+1)=
\begin{cases}
0, & T(\vec{X}(i+1)) > R \\
1, & \text{otherwise}
\end{cases}
\end{equation}

% Ecuación 21: Función de Transformación Tangente Hiperbólica T
\begin{equation}
\label{eq:21}
T(\vec{X}(i+1))=\tanh(\vec{X}(i+1))=\frac{\exp^{-\tau(\vec{X}(i+1))}-1}{\exp^{-\tau(\vec{X}(i+1))}+1}
\end{equation}

% Ecuación 22: Función de Transformación ArcTan (V-shaped) A
\begin{equation}
\label{eq:22}
A(\vec{X}(i+1))=\arctan(\vec{X}(i+1))=\left|\frac{2}{\pi}\arctan\left(\frac{\pi}{2}\vec{X}(i+1)\right)\right|
\end{equation}

% Ecuación 23: Asignación binaria para BWOA 2 (ArcTan)
\begin{equation}
\label{eq:23}
\vec{\vartheta}(i+1)=
\begin{cases}
0, & A(\vec{X}(i+1)) > R \\
1, & \text{otherwise}
\end{cases}
\end{equation}

% Ecuación 24: Función de Transformación Sigmoide Modificada S2 (S-shaped)
\begin{equation}
\label{eq:24}
S_{2}\{\vec{X}(i+1)\}=\text{sigmoid}_{2}(\vec{X}(i+1))=\frac{1}{1+e^{-10\{(\vec{X}(i+1))-0.5\}}}
\end{equation}

% Ecuación 25: Actualización binaria para BWOA 3 (Sigmoide)
\begin{equation}
\label{eq:25}
\vec{\vartheta}(i+1)=
\begin{cases}
0, & S_{2}(\vec{X}(i+1)) > R \\
1, & \text{otherwise}
\end{cases}
\end{equation}